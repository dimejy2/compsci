\documentclass{article}
\parindent 0pt
\parskip 10pt
\usepackage{amsfonts}
\begin{document}
2006 Paper 2 Question 4

Note that there was a misprint in the published version of this question. 
The sentence ``Suppose that $a$ and $b$ are co-prime, so $(a,b)=1$.'' should
appear before the definition of a natural linear combination rather than
at the beginning of Part \emph{(d)}.

(b)

Let
\begin{eqnarray*}
R &=& \{as + bt \;|\; s,t \in \mathbb{Z}\} \quad \mbox{and}\\
S &=& \{v \cdot (a,b) \;|\; v \in \mathbb{Z}\}.
\end{eqnarray*}
Then to show $R=S$ we need to show both $R \subseteq S$ and $S \subseteq R$.

First show $R \subseteq S$, i.e.\ $r \in R \Rightarrow r \in S$.

$r \in R$ so $r = as + bt$. By definition of the highest common factor,
$(a,b) \;|\; a$ and $(a,b) \;|\; b$.  So there are $a',b' \in \mathbb{Z}$
such that $a=(a,b)\,a'$ and $b=(a,b)\,b'$.
Thus $r = (a,b)\,\big(a's + b't\big) = v\cdot(a,b)$ for
$v = (a's + b't) \in \mathbb{Z}$. So $r \in S$.

Next show $S \subseteq R$, i.e.\ $r \in S \Rightarrow r \in R$.

$r \in S$ so $r = v\,(a,b)$. We have $r \in R$ if there is a solution to the
linear Diophantine equation $as + bt = r$ for some $s,t \in \mathbb{Z}$.
Such a solution exists if and only if $(a,b) \;|\; r$, which is the case
because $r = v\,(a,b)$. The solution for $s$ and $t$ can be found using the
extended Euclid's algorithm.

(Peter Robinson gives an alternative proof using well-foundedness.)

(c)

Proof by contradiction. Suppose that $ab-a-b$ can be expressed as a natural
linear combination of $a$ and $b$, i.e.\ that $ab-a-b = as+bt$ for some
$s,t \in \mathbb{N}_0$. Then
\begin{equation}\label{y2006p2q4a}
ab=a(s+1) + b(t+1).
\end{equation}
Remember that $a$ and $b$ are co-prime. Therefore in order for $b$ to divide
the right-hand side of this equation, we must have $b \;|\; (s+1)$. Now $s$
cannot be negative, so $s+1>0$, so we must have $s+1 \ge b$ because $s+1$
must be an integer multiple of $b$.

By a similar argument, we obtain $t+1 \ge a$. Substitute these two inequalities
into (\ref{y2006p2q4a}), then we get
$$ab \ge ab + ba = 2ab,$$
which is a contradiction since $a$ and $b$ are natural numbers. Therefore our
assumption must have been false. So $ab-a-b$ cannot be expressed as a natural
linear combination of $a$ and $b$.

(d)

We are given any $n \in \mathbb{N}$ with $n > ab-a-b$. Because $a$ and $b$ are
co-prime (i.e.\ $(a,b)=1$) we know that we can always express $n$ as a linear
combination $n=as+bt$ provided we allow $s$ and $t$ to range over negative
numbers too (i.e.\ $s,t \in \mathbb{Z}$). Now we want to know whether this
also holds for \emph{natural} linear combinations with $s,t \ge 0$.
Consider 4 different cases for the values of $s$ and $t$:
\begin{enumerate}
\item $s \ge 0$ and $t \ge 0$. Then this is already a natural linear combination,
and we are done.

\item $s < 0$ and $t < 0$. Then because $a$ and $b$ are natural numbers,
$n=as+bt<0$ which we have excluded by assumption.

\item $s \ge 0$ and $t < 0$. In this case pick the least $k \in \mathbb{N}$
such that $ka+t \ge 0$. (``The least'' means that if $k$ was smaller by one,
the inequality would no longer hold, i.e.\ $(k-1)a+t<0$.) Now
\begin{eqnarray*}
n &=& as + bt\\
&=& as - kab + bt + kab\\
&=& a(s-kb) + b(t+ka)
\end{eqnarray*}
We already know that $ka+t \ge 0$ so we still need to show that $s-kb \ge 0$ in
order for a natural linear combination to exist.

Recall that we are assuming $n \ge ab-a-b+1$. So
\begin{eqnarray*}
a(s-kb) + b(t+ka) &\ge& ab-a-b+1 \\
\iff a(s-kb) &\ge& ab-a-b+1-b(t+ka) \\
&=& -\big((k-1)a+t\big)b - a - b + 1 \\
&& \quad\quad\mbox{\footnotesize (We chose $k$ such that $(k-1)a+t<0$ above.} \\
&& \quad\quad\mbox{\footnotesize Thus $-((k-1)a+t) \ge 1$. So...)} \\
&\ge& b - a - b + 1 \\
&=& 1 - a \\
&>& -a
\end{eqnarray*}
The inequality $a(s-kb) > -a$ holds if $s-kb \ge 0$, as required.

\item $s < 0$ and $t \ge 0$. This case is similar to the previous one with the
variables exchanged.
\end{enumerate}
We have covered all possible cases, so we can always find a natural linear
combination under the assumptions stated above.
\end{document}
